%%%% fatec-article.tex, 2024/03/10

%% Classe de documento
\documentclass[
  a4paper,%% Tamanho de papel: a4paper, letterpaper (^), etc.
  12pt,%% Tamanho de fonte: 10pt (^), 11pt, 12pt, etc.
  english,%% Idioma secundário (penúltimo) (>)
  brazilian,%% Idioma primário (último) (>)
]{article}

%% Pacotes utilizados
\usepackage[]{fatec-article}
\Author{1}{Name={SATYRO, Gilberto S.\\ VEIGA, Pedro X. \\ SILVA, Raphael P.\\ ARAI, Vinicius C.}}

\Author{2}{Name={\{ gilberto.satyro@fatec.sp.gov.br \}\\ \{ pedro.veiga01@fatec.sp.gov.br \} \\ \{ raphael.silva130@fatec.sp.gov.br\} \\ \{ vinicius.arai@fatec.sp.gov.br \}}}

%% Definição das palavras-chaves/keywords
\Keyword{1}{Trilha Pedagógica Gamificada}{Gamified Pedagogical Trail}
\Keyword{2}{RPG (Role-Playing Game)}{RPG (Role-Playing Game)}
\Keyword{3}{Metodologias Ativas de Aprendizagem}{Active Learning Methodologies}
\Keyword{4}{Educação Integral}{Integral Education}
\Keyword{5}{Gamificação Pedagógica}{Pedagogical Gamification}
\Keyword{6}{DUA – Desenho universal da Aprendizagem}{DUA – Universal Design of Learning}
\Keyword{7}{Tecnologia Educacional}{Educational technology}

%%%% Resumo no idioma primário (brazilian)
\begin{Abstract}[brazilian]%% Idioma (brazilian ou english)
  Este artigo propõe uma análise detalhada da aplicação de uma Trilha Pedagógica Gamificada com elementos de Role-Playing Game (RPG) no contexto educacional. A pesquisa busca desenvolver estratégias inovadoras baseadas em Metodologias Ativas de Aprendizagem, promovendo um aprendizado mais significativo e inclusivo em diferentes ambientes acadêmicos. O texto destaca a relevância da Gamificação Pedagógica para engajar estudantes, desenvolver habilidades e competências, e fomentar a construção do conhecimento de forma interdisciplinar. Nesse contexto, é descrito um sistema que oferece aos estudantes estímulos por meio de uma interface de jogo eletrônico, com desafios intelectuais como perguntas e respostas, análise de dados e imagens, quebra-cabeças e combates virtuais. Os resultados obtidos alimentarão a interface do professor, apresentando dados de monitoramento e relatórios que evidenciam as habilidades desenvolvidas, recuperadas e consolidadas pelos estudantes, facilitando a elaboração de planos de aula mais coerentes e alinhados às necessidades da sala de aula. Além disso, o artigo reforça a importância de alinhar essa abordagem às diretrizes curriculares nacionais e aos objetivos de uma educação integral. Por fim, são discutidas reflexões sobre o estado da arte, metodologias similares e plataformas que implementam propostas semelhantes de gamificação pedagógica. 
\end{Abstract}

%%%% Resumo no idioma secundário (english)
\begin{Abstract}[english]%% Idioma (brazilian ou english)
  This article proposes a detailed analysis of the application of a Gamified Pedagogical Pathway with Role-Playing Game (RPG) elements in the educational context. The research aims to develop innovative strategies based on Active Learning Methodologies, promoting more meaningful and inclusive learning experiences in various academic environments. The text highlights the relevance of Pedagogical Gamification in engaging students, developing skills and competencies, and fostering interdisciplinary knowledge construction. In this context, a system is described that provides students with stimuli through a digital game interface, featuring intellectual challenges such as questions and answers, data and image analysis, puzzles, and virtual combat. The results obtained will feed the teacher's interface, presenting monitoring data and reports that highlight the skills developed, recovered, and consolidated by students, facilitating the preparation of more coherent lesson plans aligned with classroom needs. Additionally, the article emphasizes the importance of aligning this approach with national curriculum guidelines and the goals of comprehensive education. Finally, reflections on the state of the art, similar methodologies, and platforms implementing comparable gamification proposals are discussed. 
\end{Abstract}

%% Processamento de entradas (itens) do índice remissivo (makeindex)
\makeindex%

%% Arquivo(s) de referências
\addbibresource{fatec-article.bib}

%% Início do documento
\begin{document}

% Seções e subseções
%\section{Título de Seção Primária}%

%\subsection{Título de Seção Secundária}%

%\subsubsection{Título de Seção Terciária}%

%\paragraph{Título de seção quaternária}%

%\subparagraph{Título de seção quinária}%

\section*{Introdução}%
\label{sect:intro}
No cenário educacional contemporâneo, a busca por metodologias inovadoras e eficazes que promovam o engajamento dos estudantes e facilitem a construção de conhecimento tem sido uma constante. Nesse contexto, a gamificação surge como uma abordagem que integra elementos de jogos em ambientes não lúdicos, como a sala de aula, com o objetivo de tornar o processo de aprendizagem mais dinâmico, motivador e significativo \textcite{seaborn2015gamification}. Associada a essa tendência, a utilização de elementos de Role-Playing Game (RPG) se destacam como uma estratégia pedagógica promissora, capaz de estimular a participação ativa dos estudantes, o desenvolvimento de habilidades socioemocionais e a construção de conhecimento de forma transdisciplinar \textcite{hamari2014does}.

Este artigo propõe uma reflexão sobre a aplicação de uma Trilha Pedagógica Gamificada com elementos de RPG como ferramenta para potencializar o ensino e a aprendizagem, incorporando os princípios do Desenho Universal da Aprendizagem (DUA). Partindo de uma análise das demandas educacionais atuais e das possibilidades oferecidas pelas Metodologias Ativas de Aprendizagem, o presente estudo explora como a integração de elementos típicos dos jogos de interpretação de papéis podem contribuir para a promoção de uma educação mais inclusiva, democrática e alinhada com as diretrizes curriculares nacionais, bem como os Objetivos de Desenvolvimento Sustentável (ODS), que foram propostos em 2000 e adotados pelos países-membros da Organização das Nações Unidas (ONU) \textcite{de2022game}.

O ODS 4 (Educação de Qualidade), está diretamente relacionado à gamificação pedagógica, já que se concentra em garantir uma educação inclusiva, equitativa e de qualidade, promovendo oportunidades de aprendizagem ao longo da vida para todos. Por sua vez, o ODS 10 (Redução das Desigualdades), aborda a necessidade de reduzir as disparidades sociais, econômicas e políticas dentro e entre países. A gamificação pedagógica pode ser uma ferramenta poderosa para promover a inclusão e a equidade, proporcionando oportunidades de aprendizagem acessíveis para todos os alunos, independentemente de sua origem socioeconômica ou situação. Por fim, o ODS 17 (Parcerias e Meios de Implementação), destaca a importância da colaboração com outras instituições educacionais, organizações da sociedade civil e empresas. Ao trabalhar em conjunto, a gamificação pedagógica pode ampliar seu impacto e promover a implementação de todos os ODS \textcite{hamari2014does}.

A proposta apresentada neste artigo surge da necessidade de desenvolver práticas pedagógicas que considerem a diversidade de habilidades, interesses e necessidades dos estudantes. Inspirada pelo paradigma do DUA (Desenho Universal da Aprendizagem), que busca garantir o acesso ao conhecimento a todos os estudantes, independentemente de suas características individuais, sendo elaborada uma abordagem gamificada que visa não apenas promover o aprendizado dos conteúdos curriculares, mas também desenvolver habilidades cognitivas, sociais e emocionais de forma equitativa \textcite{de2022game}.

Além disso, destaca-se a importância da tecnologia educacional como um facilitador essencial nesse processo. A integração de recursos tecnológicos na Trilha Pedagógica Gamificada não apenas amplia as possibilidades de interação e personalização do ensino, mas também permite o acompanhamento individualizado do progresso dos estudantes, o compartilhamento de recursos e a criação de ambientes virtuais de aprendizagem colaborativa \textcite{questatlantis}.

Ao longo deste trabalho, serão discutidos os fundamentos teóricos da gamificação e do RPG como ferramentas educacionais, bem como serão apresentadas as possibilidades de aplicação do DUA na concepção e implementação da Trilha Pedagógica Gamificada. Além disso, serão explorados os benefícios da abordagem proposta para a promoção de uma educação inclusiva e personalizada, que atenda às necessidades de todos os estudantes. Em suma, este artigo busca contribuir para o debate sobre práticas inovadoras de ensino e aprendizagem, fornecendo subsídios teóricos e práticos para educadores interessados em explorar novas abordagens pedagógicas centradas no estudante \textcite{campbell1989heroi}.

Com base nos documentos oficiais disponibilizados pelo Ministério da Educação – Instituto Nacional de Estudos e Pesquisas Educacionais Anísio Teixeira, utilizamos os relatórios e dados do Índice de Desenvolvimento da Educação Básica (IDEB) de 2021, que consolida os resultados do Fluxo Escolar e as Médias de Desempenho nas Avaliações do Sistema de Avaliação da Educação Básica SAEB (Educação, 2022). Esta análise permite projetar e compreender os avanços e retrocessos ao longo dos ciclos da Educação Básica, desde os estudantes do 5º Ano Ensino Fundamental Anos Iniciais, 9º Ano do Ensino Fundamental Anos Finais e 3ª Série do Ensino Médio \textcite{detecnicas}.

Outra fonte importante é o Sistema de Avaliação de Rendimento Escolar do Estado de São Paulo (SARESP), que fornece dados relevantes para monitorar políticas educacionais e direcionar aprimoramentos e projetos educacionais. A comparação entre os contextos amplo (estudantes das Redes Públicas da República Federativa do Brasil) e específico (estudantes da Rede Pública do Estado de São Paulo) é fundamental para nossa proposta. O Relatório de Resultados do SARESP (2021) aborda a Evolução da Aprendizagem, detalhando as médias de proficiência por ano/série e disciplina (Tabela 1). Essa análise permite identificar quais resultados alcançaram o nível adequado de proficiência e estimar as defasagens pedagógicas quando necessário (Tabela 2) \textcite{kroneidolo}.

\begin{figure}[!h]
    \centering
    \caption{ Escala de Proficiência (Língua Portuguesa) }%
    \label{fig:tabela2}
    \includegraphics[scale=0.5]{tabela1}
    \SourceOrNote{Elaboração própria a partir do Boletim SARESP (2022)}
    \end{figure}
 
    \begin{figure}[!h]
        \centering
        \caption{ Distribuição Percentual dos Alunos da Rede Estadual de São Paulo nos Níveis de Proficiência (Língua Portuguesa) }%
        \label{fig:tabela2}
        \includegraphics[scale=0.5]{tabela2}
        \SourceOrNote{Elaboração própria a partir do Boletim SARESP (2022)}
        \end{figure}

Para aprofundar a análise dos resultados do SARESP, a Secretaria da Educação do Estado de São Paulo desenvolveu o estudo \textbf{Evolução da Aprendizagem}. Esse estudo tem como objetivo identificar as aprendizagens desejáveis e estender os Níveis de Proficiência aos anos/séries avaliados. Um destaque importante é a constatação de que estudantes da 3ª Série do Ensino Médio encerraram seus estudos com proficiência equiparável à dos estudantes do Nível Adequado do 8º Ano do Ensino Fundamental Anos Finais em Língua Portuguesa, apresentando assim, uma grande defasagem de aprendizagem. 

Portanto, as análises realizadas e a aplicação proposta terão como público-alvo os estudantes do Ensino Fundamental Anos Finais, do 9º Ano, com foco no desenvolvimento das Habilidades e Competências do Componente Arte e suas Linguagens: Artes Visuais, Dança, Música e Teatro. Essa iniciativa visa atender a um público que demonstra familiaridade com jogos eletrônicos e uma disposição maior para participar de propostas gamificadas, bem como sendo o ponto focal das Avaliações Externas de Aprendizagem ao final do ano letivo (SARESP, no Estado de São Paulo), que mensura o desenvolvimento dos estudantes deste ciclo. 

A proposta da aplicação visa envolver os estudantes em um \textbf{Universo Alternativo}, combinando elementos fantásticos com \textbf{Mitos e Lendas Nacionais}, além de períodos e eventos históricos reais, oferecendo uma nova perspectiva de aventura, chamada \textbf{Guardiões de Pindorama}. A ambientação escolhida é a região do \text{Vale do Ribeira}, representada pelos municípios abaixo, bem como os seus nomes fictícios, que serão utilizadas no jogo e o mapa da região chamada \textbf{“Nova Pindorama”} (Figura 1): 

\begin{figure}[!h]
    \centering
    \caption{ Municípios do Vale do Ribeira, Nomes Fictícios e Ambientação do Jogo (Nova Pindorama) }%
    \label{fig:figura1}
    \includegraphics[scale=0.2]{figura1}
    \SourceOrNote{Elaboração própria}
    \end{figure}

    O período escolhido como pano de fundo para a aventura é o Brasil Colônia. A justificativa para essa escolha está no fato de a cidade de \textbf{Cananéia} (1502) ser a primeira cidade colonizada no Vale do Ribeira (São Paulo/SP), seguida de \textbf{Iguape} (1538), por ser a segunda cidade mais antiga e possuir a primeira Fundição de Ouro do Estado de São Paulo. Outro ponto favorável para a escolha dessa região é a sua riqueza em recursos naturais e minerais, além de uma biodiversidade abundante de fauna e flora da \textbf{Mata Atlântica}, e uma pluralidade cultural extremamente rica, com contribuições das etnias: \textbf{Indígena} (com várias comunidades das etnias Guarani e Tupi-Guarani), \textbf{Africana} (com mais de 25 comunidades quilombolas), \textbf{Europeus} e \textbf{Orientais}. Essas culturas foram fundamentais para a formação da região.

    A intenção do jogo é fazer com que o estudante compreenda e respeite as diferenças étnicas e culturais, combata o racismo e perceba a importância das contribuições dessas culturas para a nossa sociedade.
    
    Para engajar o estudante no desenvolvimento da proposta, partimos do conceito apresentado pelo escritor e pesquisador \textbf{Joseph Campbell} (1904-1987), em sua obra \textit{O Herói de Mil Faces} \textcite{campbell1989heroi} e , que descreve as 12 etapas da chamada \textbf{Jornada do Herói}. Essa estrutura é seguida em diversos meios de entretenimento, como livros, filmes e jogos eletrônicos. Pensando dessa forma, e justificando a escolha do local, da ambientação e do período histórico, a motivação para o \textbf{Flow} (fluxo de interesse) dos estudantes não será apenas a busca pelo conhecimento, mas a tentativa de encontrar um artefato roubado, conhecido como \textbf{Ídolo de Pindorama}. Este artefato mítico, com poderes especiais, foi inspirado em um achado arqueológico real, o \textbf{Ídolo de Iguape} \textcite{kroneidolo} (Figura 2) – uma estatueta antropomorfa pré-colombiana, datada de aproximadamente 2.500 anos. O ídolo foi encontrado pelo pesquisador teuto-brasileiro \textbf{Ricardo Krone} (1862-1917), durante estudos e pesquisas no Sambaqui do Morro Grande, em \textbf{Iguape/SP} (1906). (Obs. Hoje a estatueta original encontra-se no acervo do Museu de Arqueologia e Etnologia da USP, em São Paulo)
    
    \begin{figure}[!h]
        \centering
        \caption{ \textbf{Ídolo de Iguape} (Inspiração para o enredo do Jogo) }%
        \label{fig:figura2}
        \includegraphics[scale=0.7]{figura2}
        \SourceOrNote{Acervo do MAE – Museu de Arqueologia e Etnologia / USP. }
    \end{figure}
    
    O sistema será acessado por meio de permissões específicas, que funcionarão em dispositivos de computação pessoal, sendo computador fixo (Desktop) ou computador portátil (Notebook / Laptop) disponibilizando duas interfaces com funcionalidades distintas:
    
    \begin{enumerate}
        \item \textbf{Perfil do Professor (Módulo Gestor)}: Permite ao professor cadastrar as turmas que leciona, bem como os estudantes, os componentes curriculares ministrados, as habilidades a serem trabalhadas no jogo e as perguntas pré-estruturadas para a sondagem inicial dos alunos. Além disso, possibilita o gerenciamento de bancos de questões, que serão utilizados nos desafios intelectuais ao longo do jogo.
    
        \item \textbf{Perfil do Estudante (Perfil do Usuário - Jogo)}: Dá acesso ao jogo, no qual, dependendo da etnia do personagem selecionado (\textbf{Indígena}, \textbf{Europeia} ou \textbf{Africana}), o estudante deverá responder às questões apresentadas nos desafios com base na perspectiva cultural correspondente.
    \end{enumerate}
    
    As informações sobre a evolução dos estudantes no jogo serão encaminhadas ao perfil do professor. Esses dados serão convertidos em relatórios e gráficos, permitindo análises detalhadas e intervenções pedagógicas futuras, além de auxiliar no alinhamento das atividades presenciais e/ou online.
    
    Estudos estão em andamento para determinar a melhor escolha para o desenvolvimento da aplicação móvel. Avaliando as necessidades do projeto, e as necessidades do usuário. Cada opção de plataforma de desenvolvimento será cuidadosamente considerada, levando em conta suas vantagens e desvantagens. Essa abordagem ajudará a garantir que a solução escolhida atenda efetivamente aos objetivos do projeto e às expectativas dos usuários.
    

\section*{OBJETIVO} \label{sect:obj}

O objetivo desta abordagem é permitir que os envolvidos identifiquem seus níveis de proficiência e aprofundamento em áreas específicas do conhecimento, promovendo escolhas mais assertivas para seus Projetos de Vida, seja no âmbito acadêmico ou profissional. Reconhece-se a importância de considerar tanto as afinidades individuais quanto a flexibilidade para mudanças futuras, incentivando a exploração de novas possibilidades ao longo do percurso. Para os professores, busca-se incentivar a adoção de novas metodologias, integrando o Plano de Aula ao Roteiro Pedagógico Gamificado, de forma a tornar o planejamento mais dinâmico e interdisciplinar, focado no desenvolvimento de Habilidades, Competências e Conteúdos. O processo ocorrerá em duas etapas complementares: inicialmente, atividades gamificadas estimularão o aprendizado de forma lúdica, seguidas por atividades estruturadas que consolidem os conhecimentos adquiridos, promovendo uma formação integral para estudantes e educadores. 

Quanto ao processo prático de utilização da aplicação, se dará da seguinte forma: o Professor da Sala irá acessar o sistema pelo “Módulo Gestor”, realizando um cadastro (se já não possuir um com o seu perfil), colocando seus dados pessoais; Criará um Login e Senha (para acessos futuros); Selecionará a opção de Cadastro de Turma; Incluirá todas as turmas que dará aula naquele ano letivo na unidade escolar; Incluirá seu Componente(s) Curricular(es), (OBS: o professor poderá dar aula de mais de um componente na mesma turma; Habilidades a serem desenvolvidas (Ano Corrente) e Habilidades a Serem Recuperadas/Defasadas (habilidades de Anos Anteriores que por algum motivo o estudante não tenha conseguido desenvolver no Ano correto); e por fim, realizar o Cadastro do Estudante, incluindo seus dados pessoais e finalizando o processo de cadastro desta etapa, com o Registro do Aluno (RA) como Login e uma senha Padrão genérica que poderá ser atualizada pelo estudante futuramente; Cadastro das Questões e alternativas de Respostas, que farão parte do Repositório de Perguntas utilizadas pelo Jogo. 

Já os Estudantes acessarão o Sistema pelo \textbf{Perfil de Usuário (Jogo)}, ao qual colocarão o Login (RA) e Senha Padrão de Acesso, confirmarão os seus dados pessoais e ativarão o Jogo \textbf{Guardiões de Pindorama}. Estando já no sistema do jogo, terão a primeira tela de interação, com três opções de acesso:  

\begin{itemize}[leftmargin=2em]
\item\textbf{Sair}: Permitirá que o Estudante feche a estância do jogo; 

\item\textbf{Opções}: Permitirá que o Estudante escolha se o Som ficará no Volume Mínimo, Máximo ou Desligado. Poderá escolher se a Tela de exibição do jogo ficará em modo Janela (Windows) ou Tela Cheia \textit{(Fullscreen)}e o modo de Aplicação, confirmando as seleções realizadas pelo usuário; 

\item\textbf{Jogar}: Permitirá que o Estudante escolha entre 6 personagens distintos, cada qual de uma etnia específica, com profissões próprias e status diferenciados, representados por estrelas preenchidas (valendo 5 pontos cada) e incompletas (para serem liberadas com pontos do jogo) que, que influenciarão diretamente no jogabilidade do game (OBS: As Personagens Femininas estão Bloqueadas, não sendo possível ver suas Profissões e Status. Para desbloqueá-las é necessário utilizar dinheiro do conquistado no jogo); (Tabela 3): 
\end{itemize}

\begin{figure}[!h]
    \centering
    \caption{Seleção de Personagens (Mais informações, na Secção Metodologias) }%
    \label{fig:tabela2}
    \includegraphics[scale=0.9]{tabela3}
    \SourceOrNote{Autoria Própria (2024)}
    \end{figure}

Após escolher seu Personagem, o estudante será direcionado para a Seleção de Área (Map Select), em que neste primeiro momento, ele será “carregado” em uma Região específica do Mapa, com base na Etnia escolhida e realizará um pequeno Tutorial, que explicará os comandos básicos para o usuário e terá a primeira interação com um Personagem Não Jogável (NPC – non-playable character), e responderá um conjunto de \textbf{5 perguntas} iniciais contendo \textbf{4 alternativas de resposta cada}. (OBS: estas perguntas são as mesmas para todas as etnias, onde as repostas deverão ser dadas com base na etnia em que o usuário está jogando). 

Essas perguntas servirão como base para avaliar o nível de conhecimento do estudante em relação às Habilidades e Competências do Componente de Arte. Elas subsidiarão os relatórios e as decisões sobre a organização do plano de aula e o atendimento personalizado ao estudante durante o uso do jogo, orientando as ações do professor e os próximos passos a serem desenvolvidos no jogo. 

\section*{ESTADO DA ARTE} \label{sect:estadoarte}

Esta proposta examina como a Gamificação Pedagógica pode ser implementada de forma eficaz para promover a aprendizagem das Habilidades e Competências do Componente de Arte, incentivando o desenvolvimento cognitivo, criativo e socioemocional dos estudantes do Ensino Fundamental Anos Finais no contexto da era digital. Fundamentada em metodologias ativas, a gamificação pedagógica surge como uma abordagem inovadora que alinha engajamento lúdico aos objetivos educacionais. 

Diversos estudos destacam o potencial da gamificação para fomentar motivação, engajamento e aprendizado em contextos educacionais. \textcite{hamari2014does} evidenciam que estratégias gamificadas incentivam comportamentos positivos e resultados significativos. Guias práticos, como o de \textcite{detecnicas}, fornecem orientações específicas para educadores aplicarem a gamificação de maneira eficaz. Já \textcite{seaborn2015gamification} exploram os fundamentos teóricos e aplicações práticas da gamificação, analisando seus efeitos em diversas áreas, enquanto \textcite{de2022game} investigam o uso de plataformas gamificadas na educação básica, identificando lacunas como a necessidade de estudos sobre interação, experiência do usuário e pesquisas de longo prazo. Esses estudos reforçam a relevância da gamificação como uma abordagem inovadora e eficaz no cenário educacional. 



\item \textbf{No entanto, ao analisar plataformas como: }
    \begin{itemize}[leftmargin=2em]
        \item \textbf{Quest Atlantis:} Plataforma educacional baseada em jogos desenvolvida pela University of Indiana, nos Estados Unidos. Criada por Sasha Barab, Chris Dede e Kurt Squire, com 12 contribuições de muitos outros pesquisadores e desenvolvedores, o projeto começou no final da década de 1990 e continuou a se desenvolver ao longo dos anos seguintes. O "Quest Atlantis" é um ambiente virtual de aprendizagem projetado para envolver os estudantes em experiências educacionais imersivas e interativas, permitindo que eles explorem diferentes temas e realizem missões enquanto aprendem conceitos acadêmicos; 
        \item \textbf{3DGameLab:} Plataforma de aprendizagem baseada em jogos que permite aos educadores criar cursos e atividades gamificadas para seus estudantes. Desenvolvido por Lisa Dawley e Chris Haskell na Boise State University, nos Estados Unidos, o projeto começou em 2011. Dawley e Haskell fundaram o "3D Game Lab" como um ambiente de aprendizagem inovador e envolvente para apoiar a educação baseada em jogos. A plataforma foi projetada para integrar elementos de jogos e gamificação no processo de ensino e aprendizagem, proporcionando uma experiência mais motivadora para os estudantes; 
        \item \textbf{ClassCraft:} Plataforma de gamificação para salas de aula desenvolvida por Shawn Young. O projeto começou em 2013, quando Young, um professor de física no Canadá, criou o "Classcraft" como uma maneira de tornar o aprendizado mais envolvente e motivador para seus estudantes. Desde então, a plataforma cresceu e se tornou uma ferramenta popular usada por educadores em todo o mundo para gamificar o ambiente de sala de aula e promover o engajamento dos estudantes;  
        \item \textbf{Matific:} Plataforma educacional de matemática interativa criada em 2011 para tornar o aprendizado mais envolvente. Focada no Ensino Fundamental, oferece jogos e atividades para ensinar conceitos matemáticos de forma lúdica e personalizada. A plataforma é amplamente usada em escolas ao redor do mundo. 
    \end{itemize}

Observa-se uma lacuna significativa: nenhuma dessas ferramentas aborda de forma específica o aprendizado das linguagens artísticas, como artes visuais, música, teatro e dança. 

A proposta da Trilha Pedagógica Gamificada \textit{\textbf{TPG System}} busca preencher essa lacuna ao incorporar quests e desafios diretamente relacionados ao desenvolvimento dessas competências. Por exemplo, os estudantes poderão interpretar obras de arte, criar composições musicais ou dramatizar histórias baseadas em eventos históricos, integrando conhecimento cultural e interdisciplinaridade. Além disso, elementos de RPG, como personalização de personagens e progressão baseada em conquistas, incentivam a autonomia e o pensamento estratégico dos estudantes. Essas mecânicas promovem habilidades como resolução de problemas, trabalho em equipe e empatia, essenciais para o aprendizado integral. 

Com o \textit{\textbf{TPG System}}, o professor poderá personalizar os desafios de acordo com a realidade de suas turmas ou até mesmo de estudantes individuais. O sistema contará com um banco de questões pré-estruturadas, permitindo que o professor selecione o tipo de desafio que melhor se adeque à evolução de cada estudante no jogo. Por meio de relatórios detalhados, o professor poderá ajustar o nível de dificuldade dos desafios futuros, tornando-os mais ou menos complexos, de acordo com o momento de aprendizado do estudante. Esse ajuste contínuo não apenas avalia, mas também contribui para a recuperação, o desenvolvimento, a consolidação e o aprofundamento dos conhecimentos adquiridos. 

Outra vantagem significativa do uso desta aplicação é sua capacidade de promover inclusão e acessibilidade curricular. O TPG System é projetado para atender estudantes com déficits de aprendizagem, sejam Estudantes Elegíveis da Educação Especial ou estudantes evadidos que retornaram ao ambiente escolar com lacunas de aprendizado. A plataforma permitirá que o professor identifique as necessidades específicas de cada estudante e ofereça desafios adequados à sua realidade, garantindo que todos tenham a oportunidade de progredir em seu percurso educacional de maneira inclusiva e personalizada. 

\section*{METODOLOGIA} \label{sect:metodologia}

O desenvolvimento do \textbf{TPG System} ao longo do segundo semestre de 2024 foi estruturado com o uso de diversas metodologias e ferramentas, que garantiram a eficiência do projeto. A metodologia Scrum foi adotada para identificar as necessidades reais dos usuários e garantir que o desenvolvimento das competências no componente Arte para os estudantes do Ensino Fundamental fosse bem-sucedido. Durante o processo, as necessidades dos usuários foram coletadas por meio de análise de dados de desempenho, definindo as funcionalidades principais do sistema. A equipe se organizou em ciclos de trabalho conhecidos como sprints, onde tarefas específicas foram definidas, priorizadas e executadas em um período limitado. Ao final de cada sprint, a equipe fez uma revisão, adaptando o planejamento de acordo com o feedback obtido.

A metodologia \textbf{Kanban} foi utilizada para organizar as etapas do projeto e priorizar as ações ao longo do semestre. Este sistema visual ajudou a registrar e gerenciar as tarefas do projeto, utilizando cartões e quadros para mapear o progresso. As etapas foram divididas em tópicos importantes, como a análise da devolutiva do artigo, reescrita de documentos e o acompanhamento das entregas para avaliação final, com cada tarefa sendo movida para a próxima fase à medida que avançava.

No início do desenvolvimento, o uso do \textbf{Figma} foi essencial para o design do \textbf{Módulo Gestor}, a plataforma que será utilizada pelos professores. A equipe de design trabalhou na criação de protótipos interativos, que foram testados para validar a usabilidade e a fluidez da interface. Após essa validação, a nova identidade visual foi aplicada e as telas foram refinadas, garantindo que as informações fossem organizadas de maneira clara e acessível.

Para o design dos personagens do jogo, o processo criativo foi essencial. A Hero Forge foi utilizada para modelar os personagens em 3D, criando uma base que refletisse as características desejadas. Em seguida, as imagens dos personagens foram refinadas usando o \textbf{Inkscape}, para garantir que as telas do jogo fossem estruturadas de forma intuitiva e visualmente atrativa. O Krita e o Photoshop foram utilizados para vetorização do personagem (desenhando digitalmente), pintura digital e a animação das sprites, criando sequências de movimento para os personagens, que foram cuidadosamente testadas para garantir fluidez e realismo.

Durante o desenvolvimento do jogo, as plataformas de \textbf{IA Generativa}, como \textbf{Canva}, \textbf{Leonardo IA} e \textbf{Seeart}, foram usadas para criar imagens conceituais e enriquecem o visual do jogo com gráficos inovadores baseados em prompts e imagens de referência. As imagens geradas por \textbf{IA} ajudaram a acelerar o processo criativo, fornecendo variações e novas ideias para elementos visuais do jogo.

O \textbf{MongoDB}, como banco de dados não relacional foi escolhido pela sua flexibilidade e escalabilidade. A equipe organizou o modelo de dados para suportar tanto o armazenamento das informações dos usuários quanto o acompanhamento do progresso dos estudantes no jogo. A Interface de Programação de Aplicações (API) em Node.js foi desenvolvida para conectar o banco de dados com o front-end (tela de interação do usuário junto ao sistema) do jogo e o \textbf{Módulo Gestor}, permitindo a atualização de perfis de usuários e o acompanhamento da evolução dos estudantes durante o jogo. Essa API garantiu a comunicação eficiente entre as diferentes plataformas, permitindo que as informações fossem sincronizadas em tempo real.

Para o desenvolvimento do sistema, o Visual Studio Code (VS Code) foi o ambiente de desenvolvimento integrado \textbf{(IDE)} escolhido, pois oferece uma interface simples e eficiente para codificação. A linguagem de programação Python foi usada como a principal linguagem para a codificação, sendo escolhida pela sua versatilidade e robustez. As bibliotecas \textbf{Tkinter} e \textbf{Pygame} foram utilizadas para criar as interfaces gráficas e para o desenvolvimento do jogo. \textbf{Tkinter} foi empregado para o \textbf{Módulo Gestor}, permitindo a criação de interfaces intuitivas para os professores. Já o \textbf{Pygame} foi fundamental para o desenvolvimento do jogo, garantindo uma experiência imersiva para os estudantes, com controles interativos, gráficos e animações.


\section*{RESULTADOS PRELIMINARES}\label{sect:resultados}

O desenvolvimento do TPG System ao longo do segundo semestre de 2024 alcançou avanços significativos, com base nas metodologias e ferramentas aplicadas. Os principais resultados preliminares podem ser resumidos nas seguintes áreas: 

Organização do Desenvolvimento: A aplicação das metodologias ágeis Scrum e Kanban resultou em: 

Planejamento eficaz e adaptativo: Cada sprint permitiu entregas incrementais e revisões frequentes, garantindo a evolução do projeto com base no feedback obtido. 

\begin{enumerate}[label=\arabic*)]
    \item \textbf{Organização do Desenvolvimento:} 
    A aplicação das metodologias ágeis Scrum e Kanban resultou em:
    
    \begin{itemize}[leftmargin=2em]
        \item \textbf{Planejamento eficaz e adaptativo:} Cada sprint permitiu entregas incrementais e revisões frequentes, garantindo a evolução do projeto com base no feedback obtido.
    \end{itemize}
    
    \begin{itemize}[leftmargin=2em]
        \item \textbf{Metodologia Scrum:} Conforme mencionado anteriormente, a Metodologia Scrum foi utilizada para gerenciar e priorizar os requisitos identificados durante o levantamento das necessidades dos usuários, como descrito a seguir:
    \end{itemize}
    
    \begin{itemize}[leftmargin=2em]
        \item \textbf{Contextualização:} 
        Uma Unidade Escolar necessita categorizar, qualificar, quantificar e mensurar o desenvolvimento pedagógico e cognitivo de seus estudantes do Ensino Fundamental Anos Finais, quanto ao desenvolvimento das habilidades e competências do Componente Curricular ARTE e a compreensão desta correlação com as Linguagens Artísticas: Artes Visuais, Dança, Música e Teatro. 
        Por meio desta ação, espera-se que os estudantes consigam melhorar sua compreensão, pensamento lógico e analítico, e que desta forma, possam também possuir subsídios para evoluir o seu desempenho nos Componentes: Língua Portuguesa, Matemática, Ciências, História e Geografia, que são avaliados durante as provas do SARESP (Sistema de Avaliação de Rendimento Escolar do Estado de São Paulo), avaliação realizada ao final do ano letivo para análise de desempenho escolar, pela Secretaria da Educação do Estado de São Paulo.
    \end{itemize}

    \item \textbf{Identificação das Prioridades do Cliente (Necessidades):}
    \begin{itemize}[leftmargin=2em]
        \item De um software que categoriza, qualifica, quantifica e mensura o nível de conhecimento inicial do estudante, e permite a geração de relatório diagnóstico de um determinado período;
        \item De identificar quais são as Habilidades Defasadas e/ou Não Consolidadas pelo estudante durante os Anos Anteriores, para que possa ser estruturado um Relatório de Acompanhamento de Desenvolvimento;
        \item De ter uma lista de Habilidades Defasadas e/ou Não Consolidadas pelo estudante, que necessitam serem desenvolvidas durante o ano letivo;
        \item Precisa saber se o estudante conseguiu consolidar o desenvolvimento das Habilidades Defasadas e/ou Não Consolidadas, para que possa ser ofertado os níveis de Habilidades mais complexas que devem ser desenvolvidas durante o Ano Letivo Corrente;
        \item Necessita da realização de Relatórios mensais, para acompanhamento do desempenho dos estudantes ao longo do ano letivo, que mostrem seu grau de evolução e comparativo de como estavam no início do processo e em que nível concluíram.
    \end{itemize}

    \item \textbf{Fazendo o Backlog (Levantamento de Requisitos):}
    Para determinar os requisitos necessários de uma Unidade Escolar, foi levado em consideração como:

    \begin{itemize}[leftmargin=2em]
        \item \textbf{Requisitos Funcionais:}
        \begin{itemize}
            \item \textbf{Diagnóstico Inicial do Estudante:}
            \begin{itemize}
                \item O sistema deve permitir a inserção dos dados iniciais do aluno para qualificação e quantificação do nível de conhecimento em ARTE.
                \item Permitir a geração de um relatório diagnóstico inicial com informações sobre o desempenho do aluno em diferentes habilidades e competências de ARTE.
            \end{itemize}

            \item \textbf{Identificação e Acompanhamento de Habilidades Defasadas e Não Consolidadas:}
            \begin{itemize}
                \item O sistema deve categorizar e identificar automaticamente as habilidades defasadas e não consolidadas com base no histórico do estudante.
                \item Possibilidade de gerar relatórios de habilidades defasadas e/ou não consolidadas, com uma lista detalhada para acompanhamento.
            \end{itemize}

            \item \textbf{Acompanhamento do Desenvolvimento Cognitivo e Pedagógico:}
            \begin{itemize}
                \item O sistema deve registrar o progresso do estudante, destacando o desenvolvimento em cada uma das Linguagens Artísticas (Artes Visuais, Dança, Música e Teatro).
                \item Exibir o progresso em uma interface de fácil visualização, como gráficos e tabelas, com dados históricos e comparativos.
            \end{itemize}

            \item \textbf{Consolidação de Habilidades e Identificação de Novos Níveis:}
            \begin{itemize}
                \item O sistema deve identificar e indicar quando o estudante consolida uma habilidade defasada.
                \item Sugerir automaticamente habilidades mais complexas a serem desenvolvidas no próximo período, com base nas consolidações de habilidades anteriores.
            \end{itemize}

            \item \textbf{Relatórios Mensais de Desempenho:}
            \begin{itemize}
                \item Gerar relatórios mensais do desenvolvimento do estudante, mostrando a evolução em comparação ao nível inicial.
                \item Disponibilizar opções de visualização do progresso, por competências, habilidades ou componentes curriculares.
            \end{itemize}

            \item \textbf{Exportação de Dados e Relatórios:}
            \begin{itemize}
                \item Permitir a exportação de relatórios em formatos como PDF, Excel e outros formatos comuns, para arquivamento e análise externa.
                \item Integrar uma função para o envio dos relatórios por email para os responsáveis ​​pelo acompanhamento do aluno.
            \end{itemize}
        \end{itemize}
    \end{itemize}\begin{itemize}[leftmargin=2em]
        \item \textbf{Requisitos Não Funcionais:}
        \begin{itemize}[leftmargin=2em]
            \item \textbf{Usabilidade:}
            \begin{itemize}[leftmargin=2em]
                \item A interface do sistema deve ser intuitiva e acessível para usuários com diferentes níveis de habilidade em informática.
                \item Deve seguir boas práticas de design para facilitar a navegação e a compreensão dos dados apresentados.
            \end{itemize}
    
            \item \textbf{Desempenho:}
            \begin{itemize}[leftmargin=2em]
                \item O sistema deve ser rápido e responsivo, capaz de gerar relatórios em até 5 segundos para turmas de até 50 alunos.
                \item Deve suportar muitos acessos simultâneos sem prejudicar o desempenho.
            \end{itemize}
    
            \item \textbf{Segurança:}
            \begin{itemize}[leftmargin=2em]
                \item Proteja os dados dos estudantes com criptografia e garanta o acesso apenas aos usuários autorizados.
                \item Implementar autenticação forte para o acesso ao sistema e controle de funções baseadas em papéis (administrador, professor etc.).
            \end{itemize}
    
            \item \textbf{Compatibilidade:}
            \begin{itemize}[leftmargin=2em]
                \item O sistema deve ser compatível com navegadores modernos (Chrome, Firefox, Safari) e dispositivos móveis para facilitar o acesso dos educadores e responsáveis.
                \item A versão desktop deve ser otimizada para diferentes sistemas operacionais (Windows, MacOS, Linux).
            \end{itemize}
    
            \item \textbf{Escalabilidade:}
            \begin{itemize}[leftmargin=2em]
                \item O sistema deve ser escalável para atender ao crescimento do número de estudantes e escolas sem comprometer o desempenho.
                \item A arquitetura do sistema deve permitir o fácil aumento de recursos conforme necessário.
            \end{itemize}
    
            \item \textbf{Manutenibilidade:}
            \begin{itemize}[leftmargin=2em]
                \item A implementação deve seguir boas práticas de programação para facilitar futuras atualizações e correções.
                \item O código deve ser bem documentado e modularizado para permitir manutenções e ajustes sem impactar o sistema como um todo.
            \end{itemize}
        \end{itemize}
    \end{itemize}
\end{enumerate}

\item \textbf{Realizando o Sprint Planning 1 para começar o Planejamento do Release 1.0:}
\end{enumerate}

\begin{figure}[!h]
    \centering
    \caption{Sprint Planning 1 (Planejamento da “Corrida”) – Release 1.0 (Liberação da Etapas): }%
    \label{fig:tabela4a}
    \includegraphics[scale=0.6]{tabela4a}
    \SourceOrNote{Autoria Própria (2024)}
    \end{figure}

    \begin{figure}[!h]
        \centering
        \caption{Sprint Planning 1 (Planejamento da “Corrida”) – Release 1.0 (Liberação da Etapas):  }%
        \label{fig:tabela4b}
        \includegraphics[scale=0.6]{tabela4b}
        \SourceOrNote{Autoria Própria (2024)}
        \end{figure}

\item \textbf{Realizando o Sprint Planning 2: (Tabela 5):}
    \end{enumerate}
    
    \begin{figure}[!h]
        \centering
        \caption{ Sprint Planning 2 (Planejamento da “Corrida”):  }%
        \label{fig:tabela5}
        \includegraphics[scale=0.6]{tabela5}
        \SourceOrNote{Autoria Própria (2024)}
        \end{figure}

\begin{itemize}[leftmargin=2em]
    \begin{itemize}[leftmargin=2em]
        \item O uso do Kanban otimizou a priorização de tarefas e o monitoramento do progresso, facilitando a conclusão das atividades dentro dos prazos. 
        \item \textbf{Metodologia Kanban (Tabela 6): } 
    \end{itemize}

    \begin{figure}[!h]
        \centering
        \caption{ Kanban: }%
        \label{fig:tabela6a}
        \includegraphics[scale=0.6]{tabela6a}
        \SourceOrNote{Autoria Própria (2024)}
        \end{figure}

        \begin{figure}[!h]
            \centering
            \caption{ }%
            \label{fig:tabela6b}
            \includegraphics[scale=0.6]{tabela6b}
            \SourceOrNote{Autoria Própria (2024)}
            \end{figure}

            \item \textbf{Estruturação Técnica:}
    \begin{itemize}[leftmargin=2em]
        \item API eficiente: O desenvolvimento de uma API em Node.js conectando o banco de dados ao front-end, permitindo sincronização em tempo real entre o jogo e o "Módulo Gestor".
        \item Banco de dados funcional: A implementação do MongoDB está configurada para armazenar informações dos usuários e acompanhar o progresso dos estudantes no jogo, com base no Diagrama de Banco de Dados (Tabelas 7a, 7b e 7c).
    \end{itemize}
\end{itemize}

\begin{figure}[!h]
    \centering
    \caption{Diagrama de Banco de Dados:}%
    \label{fig:tabela7a_1}
    \includegraphics[scale=0.5]{tabela7a_1}
    \SourceOrNote{Autoria Própria (2024)}
    \end{figure}

    \begin{figure}[!h]
        \centering
        \caption{Diagrama de Banco de Dados:}%
        \label{fig:tabela7a_2}
        \includegraphics[scale=0.5]{tabela7a_2}
        \SourceOrNote{Autoria Própria (2024)}
        \end{figure}

        \begin{figure}[!h]
            \centering
            \caption{Diagrama de Banco de Dados:}%
            \label{fig:tabela7a_3}
            \includegraphics[scale=0.5]{tabela7a_3}
            \SourceOrNote{Autoria Própria (2024)}
            \end{figure}

\begin{figure}[!h]
    \centering
    \caption{Diagrama de Banco de Dados:}%
    \label{fig:tabela7b_1}
    \includegraphics[scale=0.5]{tabela7b_1}
    \SourceOrNote{Autoria Própria (2024)}
    \end{figure}

    \begin{figure}[!h]
        \centering
        \caption{Diagrama de Banco de Dados:}%
        \label{fig:tabela7b_2}
        \includegraphics[scale=0.5]{tabela7b_2}
        \SourceOrNote{Autoria Própria (2024)}
        \end{figure}

        \begin{figure}[!h]
            \centering
            \caption{Diagrama de Banco de Dados:}%
            \label{fig:tabela7b_3}
            \includegraphics[scale=0.5]{tabela7b_3}
            \SourceOrNote{Autoria Própria (2024)}
            \end{figure}

    \begin{figure}[!h]
        \centering
        \caption{Diagrama de Banco de Dados:}%
        \label{fig:tabela7c_1}
        \includegraphics[scale=0.5]{tabela7c_1}
        \SourceOrNote{Autoria Própria (2024)}
        \end{figure}

        
    \begin{figure}[!h]
        \centering
        \caption{Diagrama de Banco de Dados:}%
        \label{fig:tabela7c_2}
        \includegraphics[scale=0.5]{tabela7c_2}
        \SourceOrNote{Autoria Própria (2024)}
        \end{figure}

        \clearpage
        \item \textbf{Design de utilidades:}
        \begin{itemize}[leftmargin=2em]
                \item Protótipos funcionais: O uso do Figma permitiu a validação da interface do "Módulo Gestor" (Figura 3), garantindo usabilidade e clareza.
            \end{itemize}
        \end{itemize}
        

        \begin{figure}[!h]
            \centering
            \caption{Módulo Gestão (Layout de organização das Telas do “Módulo Gestor” - Figma) }%
            \label{fig:figura3}
            \includegraphics[scale=0.3]{figura3}
            \SourceOrNote{Autoria Própria (2024)}
            \end{figure}

\item Criação visual de personagens: Personagens modelados e animados com ferramentas como Hero Forge (Figuras 4a, 4b, 4c e 4d), Inkscape (Figura 5) e Krita (Figura 6), asseguraram uma experiência visual coesa e interativa. 

\begin{figure}[!h]
    \centering
    \caption{Hero Forge (Criação da Arte Conceitual do Personagem Africano Masculino) }%
    \label{fig:figura4}
    \includegraphics[scale=0.3]{figura4}
    \SourceOrNote{Autoria Própria (2024)}
    \end{figure}

    \begin{figure}[!h]
        \centering
        \caption{Hero Forge (Criação da Arte Conceitual do Personagem Europeu Masculino) }%
        \label{fig:figura4a}
        \includegraphics[scale=0.4]{figura4a}
        \SourceOrNote{Autoria Própria (2024)}
        \end{figure}

        \begin{figure}[!h]
            \centering
            \caption{Hero Forge (Criação da Arte Conceitual e Referencias para a Animação do Personagem Indígena Masculino) }%
            \label{fig:figura4b}
            \includegraphics[scale=0.3]{figura4b}
            \SourceOrNote{Autoria Própria (2024)}
            \end{figure}

            \begin{figure}[!h]
                \centering
                \caption{Hero Forge (Criação da Arte Conceitual para as Personagens Femininas)  }%
                \label{fig:figura4c}
                \includegraphics[scale=0.3]{figura4c}
                \SourceOrNote{Autoria Própria (2024)}
                \end{figure}

                \begin{figure}[!h]
                    \centering
                    \caption{ Inkscape (Estudo de Layout para Design de Telas - “Guardiões de Pindorama”)  }%
                    \label{fig:figura5}
                    \includegraphics[scale=0.3]{figura5}
                    \SourceOrNote{Autoria Própria (2024)}
                    \end{figure}

                    \begin{figure}[!h]
                        \centering
                        \caption{ Kita (Vetorização de Personagem e estudo de Movimento para Sprites - “Guardiões de Pindorama”) }%
                        \label{fig:figura6}
                        \includegraphics[scale=0.3]{figura6}
                        \SourceOrNote{Autoria Própria (2024)}
                        \end{figure}

                        \begin{figure}[!h]
                            \centering
                            \caption{Kita (Sprites Personagem Indígena Masculino - “Guardiões de Pindorama”)  }%
                            \label{fig:figura7}
                            \includegraphics[scale=0.3]{figura7}
                            \SourceOrNote{Autoria Própria (2024)}
                            \end{figure}
\clearpage
                        
\item \textbf{Aprimoramento visual com IA (Figura 8a e 8b):}
\begin{itemize}[leftmargin=2em]
    \item A aplicação de plataformas de IA Generativa acelerou a criação de elementos gráficos, oferecendo diversidade e qualidade estética. 
    \end{itemize}  
        \end{itemize}

    \begin{figure}[!h]
        \centering
        \caption{ Leonardo IA, Canva e Seaart (Criação de Tela de “Fundo” - Apresentação do Jogo - “Guardiões de Pindorama”) }%
        \label{fig:figura8a}
        \includegraphics[scale=0.2]{figura8a}
        \SourceOrNote{Autoria Própria (2024)}
        \end{figure}
 
        \begin{figure}[!h]
            \centering
            \caption{ Leonardo IA, Canva e Seaart (Criação de Cenários - “Guardiões de Pindorama”) }%
            \label{fig:figura8b}
            \includegraphics[scale=0.2]{figura8b}
            \SourceOrNote{Autoria Própria (2024)}
            \end{figure}

            \item \textbf{Desenvolvimento do Sistema - Python:}
            \begin{itemize}[leftmargin=2em]
                \item Interface gráfica intuitiva: O "Módulo Gestor" desenvolvido com Tkinter facilita a interação dos professores com o sistema (Figuras 9a, 9b, 9c, 9d e 9e).
            \end{itemize}
            
                    
                    \begin{figure}[!h]
                        \centering
                        \caption{  Módulo Gestão (Cadastro de Usuário para acessar o sistema – Perfil Professor)  }%
                        \label{fig:figura9a}
                        \includegraphics[scale=0.2]{figura9a}
                        \SourceOrNote{Autoria Própria (2024)}
                        \end{figure}

                        \begin{figure}[!h]
                            \centering
                            \caption{ Módulo Gestão (Login de usuário para acessar o sistema)  }%
                            \label{fig:figura9b}
                            \includegraphics[scale=0.2]{figura9b}
                            \SourceOrNote{Autoria Própria (2024)}
                            \end{figure}


                            \begin{figure}[!h]
                                \centering
                                \caption{ Módulo Gestão (Opções de Cadastro de Estudantes, Turmas, Habilidades e Questões. Nesta Tela terá um Botão para realizar o download do instalador do Jogo: “Guardiões de Pindorama”) }%
                                \label{fig:figura9c}
                                \includegraphics[scale=0.2]{figura9c}
                                \SourceOrNote{Autoria Própria (2024)}
                                \end{figure}


                                \begin{figure}[!h]
                                    \centering
                                    \caption{ Módulo Gestão (Opções de Cadastro do Estudante)  }%
                                    \label{fig:figura9d}
                                    \includegraphics[scale=0.2]{figura9d}
                                    \SourceOrNote{Autoria Própria (2024)}
                                    \end{figure}

                                    \begin{figure}[!h]
                                        \centering
                                        \caption{  Módulo Gestão (Opções de Cadastro de Turmas)  }%
                                        \label{fig:figura9e}
                                        \includegraphics[scale=0.2]{figura9e}
                                        \SourceOrNote{Autoria Própria (2024)}
                                        \end{figure}
 
                                        \begin{figure}[!h]
                                            \centering
                                            \caption{  Módulo Gestão (Opções de Cadastro de Habilidades)  }%
                                            \label{fig:figura9f}
                                            \includegraphics[scale=0.2]{figura9f}
                                            \SourceOrNote{Autoria Própria (2024)}
                                            \end{figure}
 
                                            \begin{figure}[!h]
                                                \centering
                                                \caption{ Módulo Gestão (Opções de Cadastro de Questões)  }%
                                                \label{fig:figura9g}
                                                \includegraphics[scale=0.2]{figura9g}
                                                \SourceOrNote{Autoria Própria (2024)}
                                                \end{figure}
 
                                                \begin{figure}[!h]
                                                    \centering
                                                    \caption{  Módulo Gestão (Opções de acesso à relatórios de Alunos)  }%
                                                    \label{fig:figura9h}
                                                    \includegraphics[scale=0.2]{figura9h}
                                                    \SourceOrNote{Autoria Própria (2024)}
                                                    \end{figure}
                                                    \clearpage

                                                    \item \textbf{Jogo interativo: :}
                                                    \begin{itemize}[leftmargin=2em]
                                                        \item Utilizando Pygame, foram implementados gráficos, animações e controles que proporcionam uma experiência de aprendizagem envolvente, com o uso do jogo: “Guardiões de Pindorama” (Figuras 10a, 10b, 10c, 10d, 10e e 10f). 
                                                    \end{itemize}

                                                    \begin{figure}[!h]
                                                        \centering
                                                        \caption{ Pygame (Tela de Acesso ao Jogo “Guardiões de Pindorama”)   }%
                                                        \label{fig:figura10a}
                                                        \includegraphics[scale=0.2]{figura10a}
                                                        \SourceOrNote{Autoria Própria (2024)}
                                                        \end{figure}
                                
                                                        \begin{figure}[!h]
                                                            \centering
                                                            \caption{ Pygame (Tela de Menu de Opções - “Guardiões de Pindorama”)   }%
                                                            \label{fig:figura10b}
                                                            \includegraphics[scale=0.2]{figura10b}
                                                            \SourceOrNote{Autoria Própria (2024)}
                                                            \end{figure}
                                
                                
                                                            \begin{figure}[!h]
                                                                \centering
                                                                \caption{ Pygame (Tela Seleção de Personagem - “Guardiões de Pindorama”)  }%
                                                                \label{fig:figura10c}
                                                                \includegraphics[scale=0.2]{figura10c}
                                                                \SourceOrNote{Autoria Própria (2024)}
                                                                \end{figure}
                                
                                
                                                                \begin{figure}[!h]
                                                                    \centering
                                                                    \caption{ Pygame (Tela Seleção de Área (Mapa) - “Guardiões de Pindorama”)   }%
                                                                    \label{fig:figura10d}
                                                                    \includegraphics[scale=0.2]{figura10d}
                                                                    \SourceOrNote{Autoria Própria (2024)}
                                                                    \end{figure}
                                
                                                                    \begin{figure}[!h]
                                                                        \centering
                                                                        \caption{ Pygame (Tela de Fase - “Guardiões de Pindorama”)  }%
                                                                        \label{fig:figura10e}
                                                                        \includegraphics[scale=0.2]{figura10e}
                                                                        \SourceOrNote{Autoria Própria (2024)}
                                                                        \end{figure}
                                 
                                                                        \begin{figure}[!h]
                                                                            \centering
                                                                            \caption{  Pygame (Tela de Game Over - “Guardiões de Pindorama”)  }%
                                                                            \label{fig:figura10f}
                                                                            \includegraphics[scale=0.2]{figura10f}
                                                                            \SourceOrNote{Autoria Própria (2024)}
                                                                            \end{figure}
 
                                                                            \clearpage                                                                          

\section*{CONCLUSÃO}\label{sect:conclusao}

A problemática identificada reside na busca por estratégias inovadoras para promover um aprendizado mais significativo e inclusivo. Em resposta, foi proposta a implementação de uma Trilha Pedagógica Gamificada com elementos de RPG, integrando Metodologias Ativas de Aprendizagem. Essa abordagem visa engajar os estudantes, desenvolver competências e promover a construção interdisciplinar do conhecimento, em consonância com as diretrizes curriculares nacionais e os princípios de uma educação integral. 

A solução proposta contempla a criação de um sistema gamificado que permite monitoramento personalizado e decisões pedagógicas eficazes. Para sua execução, são aplicadas Metodologias Ágeis, como Scrum e Kanban, que facilitam a gestão de requisitos e tarefas. Além disso, utiliza-se uma combinação de ferramentas técnicas, incluindo desenvolvimento em Node.js e MongoDB, prototipagem no Figma e criação artística em plataformas como Hero Forge e Krita. A fase de programação envolve o uso de Python com bibliotecas como Tkinter e Pygame, visando uma aplicação funcional e envolvente. 

O desenvolvimento dessa aplicação representa um avanço significativo na integração entre tecnologia e educação, oferecendo uma ferramenta inovadora que capacita educadores e motiva estudantes. Sua estrutura intuitiva, aliada a um sistema de recompensas e flexibilidade, contribui para uma educação mais inclusiva, adaptável e alinhada às demandas do século XXI. 

\item \textbf{Proximos Passos:}
\begin{itemize}[leftmargin=2em]
    \item Inteligência Artificial (IA): Explorar o uso de IA para otimizar a análise de dados, criar questões personalizadas e adaptar o conteúdo com maior eficiência. 
    \item Melhoria Gráfica: Avaliar ferramentas como Unity, Unreal Engine e Godot para aprimorar a qualidade visual e interativa do jogo. 
    \item Acessibilidade Móvel0: Expandir o projeto para dispositivos móveis, como tablets e smartphones, garantindo maior alcance e flexibilidade aos usuários. 
\end{itemize}  
        \end{itemize}

Assim, a Trilha Pedagógica Gamificada com elementos de RPG inaugura um novo capítulo na prática educacional, promovendo um ambiente de aprendizado dinâmico, acessível e adaptado às necessidades dos estudantes. 
\clearpage

\printbibliography

%% Elementos pós-textuais (opcionais): Apêndice e Anexo
%Caso for utilizar, basta retirar o símbolo de % na frente do comando
%\input{./Extras/post-textual}

%% Fim do documento
\end{document}