Esta proposta examina como a Gamificação Pedagógica pode ser implementada de forma eficaz para promover a aprendizagem das Habilidades e Competências do Componente de Arte, incentivando o desenvolvimento cognitivo, criativo e socioemocional dos estudantes do Ensino Fundamental Anos Finais no contexto da era digital. Fundamentada em metodologias ativas, a gamificação pedagógica surge como uma abordagem inovadora que alinha engajamento lúdico aos objetivos educacionais. 

Diversos estudos destacam o potencial da gamificação para fomentar motivação, engajamento e aprendizado em contextos educacionais. \textcite{hamari2014does} evidenciam que estratégias gamificadas incentivam comportamentos positivos e resultados significativos. Guias práticos, como o de \textcite{detecnicas}, fornecem orientações específicas para educadores aplicarem a gamificação de maneira eficaz. Já \textcite{seaborn2015gamification} exploram os fundamentos teóricos e aplicações práticas da gamificação, analisando seus efeitos em diversas áreas, enquanto \textcite{de2022game} investigam o uso de plataformas gamificadas na educação básica, identificando lacunas como a necessidade de estudos sobre interação, experiência do usuário e pesquisas de longo prazo. Esses estudos reforçam a relevância da gamificação como uma abordagem inovadora e eficaz no cenário educacional. 



\item \textbf{No entanto, ao analisar plataformas como: }
    \begin{itemize}[leftmargin=2em]
        \item \textbf{Quest Atlantis:} Plataforma educacional baseada em jogos desenvolvida pela University of Indiana, nos Estados Unidos. Criada por Sasha Barab, Chris Dede e Kurt Squire, com 12 contribuições de muitos outros pesquisadores e desenvolvedores, o projeto começou no final da década de 1990 e continuou a se desenvolver ao longo dos anos seguintes. O "Quest Atlantis" é um ambiente virtual de aprendizagem projetado para envolver os estudantes em experiências educacionais imersivas e interativas, permitindo que eles explorem diferentes temas e realizem missões enquanto aprendem conceitos acadêmicos; 
        \item \textbf{3DGameLab:} Plataforma de aprendizagem baseada em jogos que permite aos educadores criar cursos e atividades gamificadas para seus estudantes. Desenvolvido por Lisa Dawley e Chris Haskell na Boise State University, nos Estados Unidos, o projeto começou em 2011. Dawley e Haskell fundaram o "3D Game Lab" como um ambiente de aprendizagem inovador e envolvente para apoiar a educação baseada em jogos. A plataforma foi projetada para integrar elementos de jogos e gamificação no processo de ensino e aprendizagem, proporcionando uma experiência mais motivadora para os estudantes; 
        \item \textbf{ClassCraft:} Plataforma de gamificação para salas de aula desenvolvida por Shawn Young. O projeto começou em 2013, quando Young, um professor de física no Canadá, criou o "Classcraft" como uma maneira de tornar o aprendizado mais envolvente e motivador para seus estudantes. Desde então, a plataforma cresceu e se tornou uma ferramenta popular usada por educadores em todo o mundo para gamificar o ambiente de sala de aula e promover o engajamento dos estudantes;  
        \item \textbf{Matific:} Plataforma educacional de matemática interativa criada em 2011 para tornar o aprendizado mais envolvente. Focada no Ensino Fundamental, oferece jogos e atividades para ensinar conceitos matemáticos de forma lúdica e personalizada. A plataforma é amplamente usada em escolas ao redor do mundo. 
    \end{itemize}

Observa-se uma lacuna significativa: nenhuma dessas ferramentas aborda de forma específica o aprendizado das linguagens artísticas, como artes visuais, música, teatro e dança. 

A proposta da Trilha Pedagógica Gamificada \textit{\textbf{TPG System}} busca preencher essa lacuna ao incorporar quests e desafios diretamente relacionados ao desenvolvimento dessas competências. Por exemplo, os estudantes poderão interpretar obras de arte, criar composições musicais ou dramatizar histórias baseadas em eventos históricos, integrando conhecimento cultural e interdisciplinaridade. Além disso, elementos de RPG, como personalização de personagens e progressão baseada em conquistas, incentivam a autonomia e o pensamento estratégico dos estudantes. Essas mecânicas promovem habilidades como resolução de problemas, trabalho em equipe e empatia, essenciais para o aprendizado integral. 

Com o \textit{\textbf{TPG System}}, o professor poderá personalizar os desafios de acordo com a realidade de suas turmas ou até mesmo de estudantes individuais. O sistema contará com um banco de questões pré-estruturadas, permitindo que o professor selecione o tipo de desafio que melhor se adeque à evolução de cada estudante no jogo. Por meio de relatórios detalhados, o professor poderá ajustar o nível de dificuldade dos desafios futuros, tornando-os mais ou menos complexos, de acordo com o momento de aprendizado do estudante. Esse ajuste contínuo não apenas avalia, mas também contribui para a recuperação, o desenvolvimento, a consolidação e o aprofundamento dos conhecimentos adquiridos. 

Outra vantagem significativa do uso desta aplicação é sua capacidade de promover inclusão e acessibilidade curricular. O TPG System é projetado para atender estudantes com déficits de aprendizagem, sejam Estudantes Elegíveis da Educação Especial ou estudantes evadidos que retornaram ao ambiente escolar com lacunas de aprendizado. A plataforma permitirá que o professor identifique as necessidades específicas de cada estudante e ofereça desafios adequados à sua realidade, garantindo que todos tenham a oportunidade de progredir em seu percurso educacional de maneira inclusiva e personalizada. 